\documentclass[10pt,landscape,a4paper]{article}

\usepackage[utf8]{inputenc}
\usepackage{lmodern}
\usepackage[T1]{fontenc}
\usepackage[english]{babel}

\usepackage[right=5mm, left=5mm, top=5mm, bottom=5mm]{geometry}
\usepackage{multicol}

\usepackage{tikz}
\usepackage[framemethod=TikZ]{mdframed}

\usepackage{amsmath}
\usepackage{amssymb}

\newenvironment{topic}[1]
{\begin{mdframed}[
    linecolor=black,
    frametitle=#1,
    frametitlebackgroundcolor=black!5,
  ]}
{\end{mdframed}}

\DeclareMathOperator{\prob}{\mathbb{P}}
\DeclareMathOperator{\E}{\mathbb{E}}
\DeclareMathOperator{\F}{\mathcal{F}}
\DeclareMathOperator{\R}{\mathbb{R}}
\DeclareMathOperator{\given}{\: \vert \:}

\DeclareMathOperator{\bernoulli}{Bernoulli}
\DeclareMathOperator{\binomial}{Binomial}
\DeclareMathOperator{\geometric}{Geom}

\newcommand{\dx}[1][x]{\,d#1}
\newcommand{\abint}{\int_a^b}
\newcommand{\inftyint}[1]{\int_{#1}^\infty}
\newcommand{\inftylim}[1]{\lim_{#1 \rightarrow \infty}}
\newcommand{\summation}[1][i = 1]{\sum_{#1}^{n}}


\pagestyle{empty}
\begin{document}
\thispagestyle{empty}

\begin{multicols}{3}

  \begin{topic}{Topology}
    Pair $(M, d)$ is metric space where $M$ is a set and $d: M \times M \rightarrow [0,\infty)$
    is a metric. It satisfies
    \begin{enumerate}
      \item $d(x, z) \leq d(x, y) + d(y, z)$
      \item $d(x, y) = d(y, x)$
      \item $d(x, y) = 0 \Leftrightarrow x = y$
    \end{enumerate}
  \end{topic}

  \begin{topic}{Probability}
    Probability space is $(\Omega, \F, \prob)$.
    Probability function is \emph{any} function $\prob : \F \rightarrow \R$
    that satisfies
    \begin{enumerate}
      \item $\forall E, 0 \leq \prob(E) \leq 1$
      \item $\prob(\Omega) = 1$
      \item $\prob \left( \bigcup\limits_{i \geq 1} E_i \right)
        = \sum\limits_{i \geq 1} \prob(E_i)$
    \end{enumerate}
    Events E and F are independent if and only if
    \[
      \prob(E \cap F) = \prob(E) \cdot \prob(F)
    \]
    Conditional probability
    \[
      \prob(E \given F) = \frac{\prob(E \cap F)}{\prob(F)}
    \]
    Law of Total Probability
    \[
      \prob(B)
      = \summation \prob(B \cap E_i)
      = \summation \prob(B \given E_i) \prob(E_i)
    \]
    Bayes' Law
    \[
      \prob(E_i \given B)
      = \frac{\prob(E_i \cap B)}{\prob(B)}
      = \frac{\prob(B \given E_i) \prob(E_i)}
      {\summation[j = 1] \prob(B \given E_j) \prob(E_j)}
    \]
    Linearity of Expectations
    \[
      \E \left[ \sum_{i = 1}^n X_i \right]
      = \sum_{i = 1}^n E[X_i]
    \]
    Jensen's Inequality. If $f$ is a convex function, then
    \[
      \E[ f(X) ] \geq f( \E[X] )
    \]
  \end{topic}

  \begin{topic}{Distributions}
    \begin{tabular}{cccc}
      Distribution & PMF & EV & Variance \\
      $\bernoulli(p)$ & $p$ & $p$ & $pq$ \\
      $\binomial(n,p)$ & $\binom{n}{k} p^k q^{n-k}$ & $np$ & $npq$ \\
      $\geometric(p)$ & $q^{n-1}p$ & $\frac{1}{p}$ & $\frac{1-p}{p^2}$ \\
    \end{tabular}
  \end{topic}

  \begin{topic}{Intermediate value}{
      \[
        f'(\xi) = \frac{f(b) - f(a)}{b - a}
      \]
      \[
        \frac{1}{b - a} \abint f(x) \dx = f(z)
      \]
      Given $h(x) \geq 0$ and $f$ is continuous then
      \[
        \frac{\abint f(x)h(x) \dx}{ \abint h(x) dx}
        = f(z)
      \]
  }\end{topic}

  \begin{topic}{Power series}{
      \[
        \Sigma a_k (x - x_0)^k
      \]
      Radius of convergence
      \[
        \inftylim{k} \left| \frac{a_k}{a_{k+1}} \right|
        \quad\text{or}\quad
        \inftylim{k} \frac{1}{\sqrt[k]{|a_k|}}
      \]
  }\end{topic}

  \begin{topic}{Taylor's theorem}{
      \[
        T_n(x; x_0) = \Sigma_{k=0}^n \frac{f^{(k)}(x_0)}{k!} (x-x_0)^k
      \]
      Integral form remainder
      \[
        R_n(x; x_0)
        = \int_{x_0}^x \frac{f^{(n+1)}(t)}{n!} (x-t)^n \dx[t]
      \]
      \[
        R_n(x; x_0) = (x - x_0)^n \varepsilon(x),
        \quad \lim_{x \rightarrow x_0} \varepsilon(x) = 0
      \]
      Cauchy form remainder
      \[
        R_n(x; x_0)
        = \frac{ f^{(n+1)}(\xi_x) }{n!} (x - \xi_x)^n (x - x_0)
      \]
      Lagrange form remainder
      \[
        R_n(x; x_0)
        = \frac{ f^{(n+1)}(\xi_x) }{ (n + 1)! } (x - x_0)^{n+1}
      \]
  }\end{topic}

\end{multicols}

\end{document}
