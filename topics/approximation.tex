\begin{definition}{$\alpha$-approximation}
  An algorithm $A$ is an $\alpha$-approximation for minimization if
  $A(I) \leq \alpha \cdot OPT(I)$ for all $I$.
\end{definition}

\begin{definition}{APX}
  Class of problems admitting a constant-factor approximation (e.g.,
  Vertex Cover, Metric TSP).
\end{definition}

\begin{definition}{PTAS (Polynomial Time Approx Scheme)}
Algorithm $A_\epsilon$ produces $(1+\epsilon)$-solution in time
$O(n^{f(1/\epsilon)})$.  Example: Euclidean TSP.
\end{definition}

\begin{definition}{FPTAS (Fully PTAS)}
  Produces $(1+\epsilon)$-solution in time polynomial in both $n$ and
  $1/\epsilon$.  Example: Knapsack.
\end{definition}

\begin{definition}{Inapproximability}
  Some problems cannot be approximated within factor $\rho$ unless
  P=NP (e.g., General TSP, Set Cover within $(1-\epsilon)\ln n$).
\end{definition}

