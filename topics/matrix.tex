\begin{definition}{LU Decomposition}
  For an invertible matrix $A$, factorization into a lower triangular
  matrix $L$ and an upper triangular matrix $U$.
  \[
    A = LU
  \]
  Used for solving linear systems $Ax=b$ and computing determinants.
\end{definition}

\begin{definition}{QR Decomposition}
  Factorization of $A$ into an orthogonal matrix $Q$ and an upper
  triangular matrix $R$.
  \[
    A = QR
  \]
  Often computed using Gram-Schmidt orthogonalization. Primarily used
  for solving linear least squares problems and as the basis for the
  QR algorithm for finding eigenvalues.
\end{definition}

\begin{theorem}{Eigen-Decomposition}
  If an $n \times n$ matrix $A$ is diagonalizable (has $n$ linearly
  independent eigenvectors), it can be decomposed:
  \[
    A = V \Lambda V^{-1}
  \]
  where $V$ is the matrix of eigenvectors and $\Lambda$ is a diagonal
  matrix containing the corresponding eigenvalues.

  If $A$ is symmetric, then $V$ is an orthogonal matrix.
\end{theorem}

\begin{theorem}{Singular Value Decomposition (SVD)}
  Any $m \times n$ matrix $A$ can be decomposed as:
  \[
    A = U \Sigma V^T
  \]
  where $U$ is $m \times m$ orthogonal, $V$ is $n \times n$
  orthogonal, and $\Sigma$ is $m \times n$ diagonal with non-negative
  real values (singular values $\sigma_i$).

  Used for dimensionality reduction (PCA), low-rank approximation,
  and finding pseudo-inverses.
\end{theorem}
